\title{Syllabus for Elementary Statistics: MATH 080 Summer 2020}
\author{Dr. Jordan Hanson - Whittier College Dept. of Physics and Astronomy}
\date{\today}
\documentclass[10pt]{article}
\usepackage[a4paper, total={18cm, 27cm}]{geometry}
\usepackage{outlines}
\usepackage{hyperref}
\begin{document}
\maketitle

\begin{abstract}
This course will follow the standard methodology of elementary statistics taught regularly during Fall and Spring semesters. Concepts of descriptive statistics will be presented, including descriptive measures, probability concepts, discrete random variables, and the normal distribution. Concepts of inferential statistics will also be presented, including sampling distributions, confidence intervals, hypothesis testing, chi-squared procedures, and linear regression. This version of the course will provide interactive learning techniques designed to augment the online learning experience, and the course textbook will be free.  Time-permitting, the use of computer programming tools to help visualize examples will be covered.
\end{abstract}
\noindent
\textit{\textbf{Pre-requisites}: C- or higher in MATH 076 or MATH 079 or 2 or better on the Math Placement Exam } \\
\textit{\textbf{Course credits, Liberal Arts Categorization}: 3 Credits, None} \\
\textit{\textbf{Regular course hours}: Monday through Friday, 9:00-12:00.  Synchronous time: 1.0-1.5 hours, asynchronous time: 1.0-1.5 hours.  Synchronous time corresponds to instruction via \textbf{Discord}, asynchronous time corresponds to engaging with pre-recorded content and thinking through content that prepares you for the homework.} \\
\textit{\textbf{Instructor contact information}: jhanson2@whittier.edu, \textbf{Discord}: 918particle\#5083} \\
\textit{\textbf{Office hours}: Fridays from 8:00-12:00, via \textbf{Discord}.} \\
\textit{\textbf{Attendance/Absence}: Students are required to log in for the morning Zoom sessions, however the asynchronous content may be digested on the students' individual schedules}.\\ 
\textit{\textbf{Late work policy}: Late work is generally not accepted, but is left to the discretion of the instructor.} \\
\textit{\textbf{Text}: Introductory Statistics - \url{https://openstax.org/details/books/introductory-statistics} The text is open-access, and therefore free online, and via the OpenStax app on iOS and Android.  High-quality PDF is available.} \\
\textit{\textbf{Grading}: There will be one midterm, homework assignments, and daily warm-up quizzes.  Homework is from the open-access textbook, submitted via Moodle.  Finally, there will be a self-designed project and presentation given at the end of the term.  Daily warm-up quizzes are graded for completion, like taking attendance. See Table \ref{tab:grades} for grading percentages.} \\

\begin{table}[h]
\centering
\begin{tabular}{| c | c |}
\hline
Item & Percentage \\ \hline \hline
Warm ups & 20 \% \\ \hline
Homework & 30 \% \\ \hline
Midterm & 25 \% \\ \hline
Final Project + Presentation & 25\% \\ \hline
\end{tabular}
\caption{\label{tab:grades} (Left) These are the grade settings with the final exam included. (Right) These are the grade settings without the final exam.  The final exam is optional.}
\end{table}
\vspace{0.5cm}
\noindent
\textit{\textbf{Grade Settings}: $\geq 60\%, <70\%$ = D, $\geq 70\%, <80\%$ = C, $\geq 80\%, <90\%$ = B, $\geq 90\%, <100\%$ = A.  \\ Pluses and minuses: 0-3\% minus, 3\%-6\% straight, 6\%-10\% plus (e.g. 79\% = C+, 91\% = A-)} \\
\textit{\textbf{Statement on Disability Services}: Whittier College is committed to make learning experiences as accessible as possible. If you experience physical or academic barriers due to a disability, you are encouraged to contact Student Disability Services (SDS) to discuss options. To learn more about academic accommodations, please email disabilityservices@whittier.edu.} \\
\textit{\textbf{Mental Health Resources:} Counseling services for enrolled students are offered at no charge and will be offered remotely during periods of remote learning. When we return to campus, in person and remote/telehealth services will be offered. Schedule an appointment by emailing counselingcenter@whittier.edu  or by phone, 562-907-4239 (between 8am – 5pm; M-F). For support after 5pm, you may call the After-Hours RN Telephone Advice Line at 562.454.4548 (press option 1); for mental health emergencies contact Campus Safety at 562-907-4211; Digital mental health platform at \url{https://you.whittier.edu/}} \\
\textit{\textbf{Academic Honesty Policy}: \url{http://www.whittier.edu/academics/academichonesty}}

\clearpage

\textit{\textbf{Course Objectives}:}
\begin{itemize}
\item To practice written and oral expression of technical ideas.
\item To solve word problems pertaining to statistics.
\item To learn to describe data sets and populations with statistical parameters.
\item To learn to predict the behavior of data sets and populations with statistical tools.
\item To perform experiments and data analysis, and the communication of results.
\end{itemize}

\textit{\textbf{Course Outline}:}
\begin{outline}[enumerate]
\1 \textbf{Unit 0}: Introductory concepts
\begin{enumerate}
\item Topics from Chapter 1: 1.1, 1.2, 1.3
\begin{itemize}
\item What is a statistic?
\item Probability examples
\item Data and sampling
\end{itemize}
\item Topics from Chapter 2: 2.1 - 2.4, 2.5 - 2.8
\begin{itemize}
\item Data visualization
\item Location of the data in numerical space
\end{itemize}
\item Topics from Chapter 3: 3.1, 3.2, 3.3
\begin{itemize}
\item Two rules of probability
\end{itemize}
\end{enumerate}
\1 \textbf{Unit 1}: Discrete Random Variables and the Normal Distribution
\begin{enumerate}
\item Topics from Chapter 4: 4.1 - 4.4
\begin{itemize}
\item Discrete random variables, averages and standard deviation
\item Distributions of discrete random variables: binomial and geometric
\end{itemize}
\item Topics from Chapter 6: 6.1 - 6.4
\begin{itemize}
\item The normal and standard normal distributions
\item Using normal distributions
\end{itemize}
\end{enumerate}
\1 \textbf{Unit 2:} Applications of descriptive statistics
\begin{enumerate}
\item Applications to baseball (Moneyball)
\item Applications to stock exchanges (Wall Street)
\item Applications to COVID-19 (Normal distributions and standard deviations)
\item Central limit theorem: Chapters 7.1 and 7.2, with examples
\end{enumerate}
\1 \textbf{First midterm, July 31st, 2020.} The first midterm will cover units 0, 1, and 2. Emphasis on 0 and 1.
\1 \textbf{Unit 3:} Confidence intervals and hypothesis testing
\begin{enumerate}
\item Topics from Chapter 8: 8.1 - 8.4
\begin{itemize}
\item Confidence intervals
\item Data interpretation
\end{itemize}
\item Topics from Chapter 9: 9.1 - 9.3, 9.6
\begin{itemize}
\item Rejection of the Null hypothesis
\item Types of error
\item Underlying distributions
\end{itemize}
\end{enumerate}
\1 \textbf{Unit 4:} The Chi-squared test and linear regression
\begin{enumerate}
\item Topics from Chapter 11: 11.1 - 11.4
\begin{itemize}
\item How to perform Chi-squared tests
\item Testing variables' independence
\item Significance
\item Special topic: a function's fit to time-dependent data
\end{itemize} 
\item Topics from Chapter 12: 12.1 - 12.6
\begin{itemize}
\item Linear regression
\item Data linearization and visualization
\item Correlation and the correlation coefficient
\item \textbf{Final project:} (1) Location and measurement of data sample (2) Data cleaning and linearization (3) Regression analysis (4) Extrapolation
\end{itemize}
\item \textbf{Final project presentations} (last week of class: August 10th through August 14th)
\end{enumerate}
\1 \textit{Final Project and Presentation details:}
Students will perform data collection and analysis on a topic of their choice, with the goal of establishing a linear correlation.  Often times this portion of the project is more art than science, since many experiments do not turn out a linear dependence.  Students must submit a \textit{project summary} by July 31st, 2020 so that the instructor may modify it or make suggestions.  The data should be collected, along with statistical errors on the measurements.  Students will then perform a linear regression on the data, and attempt to make a sound statistical prediction using the regression.  The final presentation should be 7-12 slides in PDF or PowerPoint format, which will be presented remotely to the class via Zoom in the final week of class.  In addition to the technical correctness of the conclusions, students will be assessed on the clarity of the presentation.  \textbf{As such, the instructor will provide practice sessions, training, and advice on presentation of the topic by appointment with the students.}
\end{outline}
\end{document}
