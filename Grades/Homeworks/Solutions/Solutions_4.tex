\documentclass{article}
\usepackage{graphicx}
\usepackage{url}
\usepackage[margin=1.5cm]{geometry}
\usepackage{amsmath}

\begin{document}

\title{Solutions to Homework 4}
\author{Prof. Jordan C. Hanson}

\maketitle

Exercises, Chapter 7: Exercises 67, 68, 69, 71, and Chapter 8: Exercises 95, 96, 97

\begin{enumerate}
\item Chapter 7, Exercise 67: (a) True, by the central limit theorem the sample means approach the population mean. (b) True, by the central limit theorem the means have to be normally distributed. (c) False, there is a factor of $\sqrt{n}$ from the central limit theorem.
\item Chapter 7, Exercise 68: (a) $N(36,10/\sqrt{16}) = N(36,2.5)$ (b) $P(\bar{X} > 5)$ is the same as the probability that $\bar{X}$ is measured above 5, which is more than 12 standard deviations from the mean.  Thus, there is almost a 100 percent probability that $\bar{X} > 5$. (c) It turns out that -0.67 standard deviations below the mean of a normal distribution is the location of Q1.  This is a location of 34.3 calories (36 minus 0.67 times 2.5).
\item Chapter 7, Exercise 69: (a) X is the typical salary per year in USD. (b) $\bar{X}$ is the average salary per year drawn from the wedge distribution. (c) $\bar{X} = N(2000,8000/\sqrt{1000}) = N(2000,253)$. (d) In the wedge distribution described above, the mean is close to zero but there is a long tail. (e) The former is closer to the mean, and the distribution of the mean is normal.
\item Chapter 7, Exercise 71: B 
\item Chapter 8, Exercise 95: (a) i) 71 ii) 2.8 iii) 48 (b) X is the distribution of heights, and $\bar{X}$ is the mean of that distribution. (c) If we're measuring the mean, we use the normal distribution.  However, the small sample size would require the Student-t distribution rather than the normal if we were computing t-scores. (d) The standard error in the mean is 0.4 inches, so the confidence interval is $[71 - 2 \times 0.4, 71 + 2\times 0.4] = [70.2, 71.8]$ inches. (e) The level of confidence will increase, because the same confidence interval would correspond to more standard deviations from the mean.
\item Chapter 8, Exercise 96: (a) $X$ is the length of conferences in days and $\bar{X}$ is the mean of the sample. (b) The normal distribution will describe the mean of the sample, following the CLT. (c) $[3.66,4.22]$ days
\item Chapter 8, Exercise 97: (a) i) 23.6 hours ii) 7.0 hours iii) 100 (b) $X$ is the number of hours to complete the tax forms and $\bar{X}$ is the sample mean. (c) Normal distribution, following central limit theorem with a large $n$ value. (d) 1.645 standard deviations above and below the mean corresponds to the 90 CL.  Thus, $[23.6-1.645*7/\sqrt{100}, 23.6+1.645*7/\sqrt{100}] = [22.4, 24.8]$ hours.
\end{enumerate}

\end{document}
