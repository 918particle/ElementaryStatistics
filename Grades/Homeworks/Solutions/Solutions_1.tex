\documentclass{article}
\usepackage{graphicx}
\usepackage{url}
\usepackage[margin=1.5cm]{geometry}
\usepackage{amsmath}

\begin{document}

\title{Solutions to Homework 1}
\author{Prof. Jordan C. Hanson}

\maketitle

Exercises: 42, 44, 46, 51, 52, 54, 56, 58, 60, 66, 76, 90

\begin{enumerate}
\item Exercise 42: 
\begin{itemize}
\item a) The population is the set of all clients
\item b) The sample is a subset of the population actually measured
\item c) The parameter is the average of the population: the mean time spent exercising
\item d) The statistic is the sample mean of times spent at the gym
\item e) The variable is the time per trip to the gym
\item f) The data is a list of times spent at the gym, forming the sample
\end{itemize}
\item Exercise 44: 
\begin{itemize}
\item a) The population is all of the patients who have had heart attacks
\item b) The sample is the subset of the population actually measured who have had heart attacks
\item c) The parameter is the mean time of recovery for the whole population 
\item d) The statistic is the mean recovery time found in the sample
\item e) The variable is the amount of time needed for recovery
\item f) The data is the list of recovery times for each patient in the sample
\end{itemize}
\item Exercise 46: 
\begin{itemize}
\item a) All voter's in the politicians district
\item b) A random selection of voters in the district
\item c) The proportion of voters in this district who think the politician is doing a good job
\item d) The same propotion as (c) but for the sample
\item e) The number of voters in the sample who check yes divided by the total number
\item f) Qualitative or categorical data
\end{itemize}
\item Exercise 51: The correct choice is \textbf{a}, a variable.
\item Exercise 52: The correct choice is \textbf{c}, a statistic.
\item Exercise 54: Percent body fat is an example of quantitative continuous data (e.g. 15.1 percent).
\item Exercise 56: Time in line is an example of quantitative continuous data (e.g. 7 minutes).
\item Exercise 58: Most watched television show is an example of qualitative or categorical data
\item Exercise 60: Distance is quantitative continuous (e.g. 4.5 km)
\item Exercise 66: \textbf{One example given here:}
\textit{Use simple random sample to choose 25 colleges in the state. Use all statistics classes from each college. List all the colleges together with a two-digit number (\url{http://en.wikipedia.org/wiki/List_of_colleges_and_universities_in_California}) Use a random number generator to pick 25.}
\item 
\begin{itemize}
\item a) During the Great Depression, many people could not afford the items that would have placed them on those lists, so the high N value incurred sampling error.
\item b) Samples that are too small can have sampling error or bias
\item c) Sampling error
\item d) Stratified sampling
\end{itemize}
\item 
\begin{itemize}
\item a) 4 percent
\item b) 13 percent
\item c) Not necessarily.  The N values could be different, but moreoever, the population demographics could be different.
\end{itemize}
\end{enumerate}

\end{document}
