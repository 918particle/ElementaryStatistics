\documentclass{article}
\usepackage{graphicx}
\usepackage{url}
\usepackage[margin=1.5cm]{geometry}
\usepackage{amsmath}

\begin{document}

\title{Solutions to Homework 3}
\author{Prof. Jordan C. Hanson}

\maketitle

Exercises, Chapter 3: 67, 82, 84, 85, 86

\begin{enumerate}
\item Exercise 67: (a) You can't have probability greater than 130 percent, so we shouldn't add these numbers.  It's not just Saturday OR Sunday, because we don't know the probability of it raining Saturday AND Sunday. (b) These events are not mutually exclusive.  If it's a home run, then it's a successful hit.
\item Exercise 82: (a) 00, 0, and 1-36. (b) 18/38 = 9/19 (less than half). (c) 12/38 = 6/19 (about 30 percent). (d) The 00 and 0 squares don't count.  So it's back to 9/19. (e) No, because of the 0 and 00 squares. (f) For example, even and odd cannot both occur. (g) No.
\item Exercise 84: (a) 9/19 (b) 6/19 (c) 18/38 = 9/19 (d) 9/19 because there are still 18 numbers in [19-36] (e) 6/19 (f) 9/19
\item Exercise 85: (a) G1 G2 G3 G4 G5 Y1 Y2 Y3 (b) 5/8 (c) 2/3 (d) 1/4 (e) 6/8 = 3/4 (you can't double-count cards). (f) No, because P(G AND E) > 0.
\item Exercise 86: (a) (1,1), (1,2), ... (1,6), (2,1), (2,2), ... (6,6) (there are 36 combinations). (b) (1/3) x (1/2) = (1/6). (c) 21/36 (there are 21 ways to reach a 7 or lower). (d) 1/7 (e) No. (f) These events are dependent. Knowing that one event occurs does change the probability that the other occurs. If you know what the sum of the dice is at most 7 (event B), then there are fewer ways for event A to occur.
\end{enumerate}

\end{document}
