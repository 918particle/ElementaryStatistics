\documentclass{article}
\usepackage{url}
\usepackage[margin=1.5cm]{geometry}
\usepackage{amsmath}

\begin{document}

\title{Laboratory Activity 4 for Math 080: Topics from Chapter 2}
\author{Jordan C. Hanson}
\maketitle

\section{Introduction}

Please watch the video entitled ``Quartiles and Percentiles'' on Moodle, and apply the concepts of Unit 0 in the following exercise.  The data for this lab activity is stored on Moodle under ``Class GPA Distribution.''

\section{GPA Distributions}

\begin{enumerate}
\item A sample of $N = 100$ GPA results has been collected via random sampling from a population of 10,000 students from Whittier College over the past few years.  Please download the data from Moodle and open it in Microsoft Excel or LibreOffice Calc.  Although it has a file extension of ``csv,'' the raw data has no commas and can be imported using the default settings.
\item Calculate the mean and median GPA of the sample. (a) Are the results the same?  If they are different, list some potential reasons. (b) Calculate the standard deviation of the data set.
\item Create a histogram of the data set, and identify on your graph the mean and standard deviation.  (a) Is there any indication that these two numbers (the mean and standard deviation) might not fully locate or describe the position of the data?
\item Your task is to now ``cut'' the data into two subsets that might be better described with two means and two standard deviations.  (a) Sort the data into ascending order.  (b) Use your judgement to identify a cut location and separate the data into two subsets.  The cut location, for example, could correspond to a quartile or certain percentile.
\item Recompute the mean and standard deviation, and create two new histograms, for each subsample.  Do the results make more sense than as a single histogram?  Why or why not?
\end{enumerate}



\end{document}