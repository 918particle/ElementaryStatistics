\documentclass{article}
\usepackage{graphicx}
\usepackage[margin=1.5cm]{geometry}
\usepackage{amsmath}

\begin{document}

\title{Laboratory Activity 3 for Math 080: Topics from Chapter 1}
\author{Jordan C. Hanson}
\maketitle

\section{Introduction}

Please watch the video entitled ``Two-dimensional histograms'' on Moodle, and repeat the calculation shown there, according to the parameters below.  Perform your calculation in Excel or Calc.  \textbf{The goal is to fill in the table below, using your results.}

\section{Questions}

\begin{enumerate}
\item Calcualte the relative frequencies of the numbers of classes as done in the video.
\item Remove the final \textit{overflow} column, as done in the video.
\item Create a 2D histogram from the relative frequencies that contains 5 rows and 3 columns.  This is equivalent to filling in the following table:
\begin{table}[ht]
\centering
\begin{tabular}{| c | c | c | c |}
\hline
\hline
(Years/Sizes) & 0-19 & 20-39 & 40-99 \\ \hline
2010-12 &  &  &  \\ \hline
2012-13 &  &  &  \\ \hline
2014-15 &  &  &  \\ \hline
2015-17 &  &  &  \\ \hline
2018-19 &  &  &  \\ \hline
\hline
\end{tabular}
\end{table}
\item Do you identify any trends?  If so, what are they?  (Base your arguments on the histogram you've constructed from the original data).--
\end{enumerate}

\end{document}