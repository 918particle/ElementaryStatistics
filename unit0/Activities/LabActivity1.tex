\documentclass{article}
\usepackage{graphicx}
\usepackage[margin=1.5cm]{geometry}
\usepackage{amsmath}
\usepackage{url}

\begin{document}

\title{Laboratory Activity 1 for Math 080: Topics from Chapter 1}
\author{Jordan C. Hanson}
\maketitle

\section{Introduction}

Steven Levitt and Stephen J. Dubner are two economic researchers and authors of the book \textit{Freakonomics}.  They recorded a podcast and posted it to the Freakonomics website here: \\ \\
\url{https://freakonomics.com/podcast/how-much-do-we-really-care-about-children-ep-447}.

\section{Conceptual Questions about the Podcast}

In the first segment about child fertility and car seats, identify one way the following concepts arise in the discussion:
\begin{itemize}
\item \textbf{Probability:}
\item \textbf{Population:}
\item \textbf{Sample:}
\end{itemize}
In the second segment about child safety and car seats, identify one way the following concepts arise in the discussion:
\begin{itemize}
\item \textbf{Statistic:}
\item \textbf{Parameter:}
\item \textbf{Representative sample:}
\item \textbf{Variable, data:}
\end{itemize}

\section{Quantitative Questions about the Podcast}

In the first segment about child fertility and car seats, the \textit{number of children born per woman} is mentioned as a statistic used to study population growth.  Suppose we use anonymized hospital data to determine the number of children born to 10 different women (Tab. \ref{tab:1}). (a) What is the average number of children per woman in Tab. \ref{tab:1}? (b) What is the average children per woman for each income bracket? (c) What can be done to improve the sample in Tab. \ref{tab:1}, in order to better capture population trends?

\begin{table}
\small
\centering
\begin{tabular}{c | c | c }
Age of mother & Number of children & Income bracket \\ \hline
30 & 3 & A \\ \hline
30 & 1 & B \\ \hline
26 & 1 & A \\ \hline
34 & 3 & D \\ \hline
32 & 5 & C \\ \hline
32 & 2 & C \\ \hline
16 & 1 & D \\ \hline
34 & 2 & A \\ \hline
26 & 2 & A \\ \hline
23 & 1 & A \\ \hline
\end{tabular}
\caption{\label{tab:1} (Left column): age of the mother when the most recent child is born. (Middle column): Cumulative number of children to date. (Right column): Income bracket, with A representing lowest income, and D representing highest income.}
\end{table}

\end{document}