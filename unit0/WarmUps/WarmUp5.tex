\documentclass{article}
\usepackage{graphicx}
\usepackage[margin=1.5cm]{geometry}
\usepackage{amsmath}

\begin{document}

\title{Warm-Up Day 5}
\author{Prof. Jordan C. Hanson}

\maketitle

\section{Formula Area}

\begin{itemize}
\item \textit{Median}: The value that separates the lower half of a sorted list of data from the upper half.  It does not have to be one of the data values.
\item \textit{Quartiles}: numbers that divide the data into fourths, or equal frequency in four bins.
\item \textit{IQR}: the inter-quartile range is $Q_3 - Q_1$, or the third quartile minus the first quartile.
\end{itemize}

\section{Locating the Center of the Data}

\begin{enumerate}
\item Sort the list of GPAs in Tab. \ref{tab:gpa} into \textit{ascending order}. What is the median of the data? \\ \vspace{2cm}
\item Group the data into \textit{quartiles}, so that one fourth of the sorted data is in each bin.  What is the IQR?  Are there any outlying data points? \\ \vspace{2cm}
\item Calculate the \textit{cumulative relative frequencies} of the sorted data.  Below what number does 90 percent of the data appear? (This leads into a discussion of the percentiles).
\end{enumerate}

\begin{table}[hb]
\centering
\begin{tabular}{| c | c |}
\hline \hline
Student ID & GPA \\ \hline
A & 3.1 \\ \hline
B & 3.7 \\ \hline
C & 2.9 \\ \hline
D & 3.1 \\ \hline
E & 3.3 \\ \hline
F & 3.6 \\ \hline
G & 2.7 \\ \hline
H & 3.9 \\ \hline
I & 2.9 \\ \hline
J & 3.3 \\ \hline
K & 3.6 \\ \hline
I & 4.0 \\ \hline
\hline
\end{tabular}
\caption{\label{tab:gpa} A table of student GPA's for a single semester.}
\end{table}

\end{document}
