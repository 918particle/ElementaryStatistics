\documentclass{article}
\usepackage{graphicx}
\usepackage[margin=1.5cm]{geometry}
\usepackage{amsmath}

\begin{document}

\title{Warm-Up for Day 6}
\author{Prof. Jordan C. Hanson}

\maketitle

\section{Formula Area}

\begin{itemize}
\item \textit{Standard deviation, s}:
\begin{equation}
s^2 = \frac{1}{N-1}\sum_{i=1}^{N} (x_i - \bar{x})^2
\end{equation}
\item In the previous equation, $\bar{x}$ is the mean of the sample of size $n$.
\end{itemize}

\section{Understanding the Spread of Data}

\begin{enumerate}
\item The standard deviation describes how far a typical piece of data is from the mean.  First, calculate the mean of the stock prices in Tab. \ref{tab:scores} using Calc or Excel\footnote{By the way, an analysis of Yahoo Finance historical stock prices would make an excellent final project.}. \\ \vspace{1cm}
\item Next, calculate the difference between each data point and the mean.  Then, square that difference to obtain the deviation-squared. \\ \vspace{1cm}
\item Finally, sum the deviations-squared, and divide by $N-1$, where $N$ is the number of stocks in the sample.  What result do you find?  If you create a histogram of the data, you can see the meaning of the standard deviation visually.
\end{enumerate}

\begin{table}[hb]
\centering
\begin{tabular}{| c | c |}
\hline \hline
Stock Label & Price (USD) \\ \hline
A & 46.4 \\ \hline
B & 57.2 \\ \hline
C & 38.2 \\ \hline
D & 48.3 \\ \hline
E & 33.2 \\ \hline
F & 56.2 \\ \hline
G & 38.3 \\ \hline
H & 45.3 \\ \hline
I & 41.1 \\ \hline
J & 53.2 \\ \hline
K & 51.9 \\ \hline
L & 38.4 \\ \hline
M & 60.7 \\ \hline
N & 49.8 \\ \hline
O & 46.2\\ \hline
\hline
\end{tabular}
\caption{\label{tab:scores} A listing of stock prices in USD for today.}
\end{table}

\end{document}