\documentclass{article}
\usepackage{graphicx}
\usepackage[margin=1.5cm]{geometry}
\usepackage{amsmath}

\begin{document}

\title{Warm-Up 3}
\author{Prof. Jordan C. Hanson}

\maketitle

\section{Frequencies, Relative Frequencies}
The data in Tab. \ref{tab:data} refer to the subjects' assessment on a scale of 1 to 10 of the hotness of their favorite chili variety.
\begin{enumerate}
\item Suppose you have a dataset given by Tab. \ref{tab:data}.  Determine each of the following:
\begin{itemize}
\item The frequency and relative frequency of ``serrano'' as the favorite chili of the subject.
\item The frequency and relative frequency of haba\~{n}ero as the favorite chili of the subject.
\item What is the mean ``hotness'' rating of all the chiles?
\item What is the mean ``hotness'' rating of just the jalape\~{n}os?
\end{itemize}
\begin{table}[ht]
\centering
\begin{tabular}{| c | c | c |}
\hline 
\hline
Person & Chili & Hotness rating \\ \hline
Juan & Serrano & 5 \\ \hline
Gilberto & Jalape\~{n}o & 4 \\ \hline
Catherine & Arbol & 7 \\ \hline
Jordan & Serrano & 7 \\ \hline
Chris & Pablano & 2 \\ \hline
Rosa & Arbol & 8 \\ \hline
William & Jalape\~{n}o & 9 \\ \hline
Lupe & Haba\~{n}ero & 9 \\ \hline
Jennifer & Serrano & 8 \\ \hline
Jasmine & Arbol & 6 \\ \hline
Jose & Serrano & 6 \\ \hline
Roberto & Serrano & 10 \\ \hline
Kevin & Jalape\~{n}o & 4 \\ \hline
Sarah &  Jalape\~{n}o & 9 \\ \hline
\hline
\end{tabular}
\caption{\label{tab:data} A tabulation of the perceived hotness on a scale of 1 to 10 for subject's favorite chili.  \textit{I don't know what you call those yellow ones...Guerrero? Just yellow?}}
\end{table}
\item \textbf{Draw a histogram} of the hotness rating of all the chiles below.  Choose simple bins: 1, 2, 3, ... 10.  For each bin on the x-axis, assign a bar that has the same height as the number of times that bin value appears in the data set.
\end{enumerate}

\end{document}
