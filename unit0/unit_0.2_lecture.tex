\documentclass{beamer}
\usetheme{metropolis}
\usepackage{graphicx}
\usepackage{amsmath}
\usepackage{tcolorbox}
\title{Elementary Statistics: Math 080}
\author{Jordan Hanson}
\institute{Whittier College Department of Physics and Astronomy}

\begin{document}
\maketitle

\begin{frame}{Unit 0 Outline}
\begin{enumerate}
\item Topics from Chapter 1: 1.1, 1.2, 1.3
\begin{itemize}
\item What is a statistic?
\item Probability examples
\item Data and sampling
\end{itemize}
\item Topics from Chapter 2: 2.1 - 2.4, 2.5 - 2.8
\begin{itemize}
\item Data visualization
\item Location of the data in numerical space
\end{itemize}
\item Topics from Chapter 3: 3.1, 3.2, 3.3
\begin{itemize}
\item Two rules of probability
\end{itemize}
\end{enumerate}
\end{frame}

\section{Topics from Chapter 2}

\begin{frame}[fragile]{Stemplots}
\small
Useful for numbers like \textit{grades}. Most significant digit is the category.
\begin{table}
\begin{tabular}{| c | c |}
\hline
\hline
Stem & Leaves \\ \hline
0 &   \\ \hline
1 &   \\ \hline
2 &   \\ \hline
3 &   \\ \hline
4 & [3.0] \\ \hline
5 & [6.0] \\ \hline
6 & [7.0, 9.0] \\ \hline
7 & [8.0, 0.0, 8.0, 1.0, 2.0, 5.0, 7.0] \\ \hline
8 & [8.0, 3.0, 4.0, 6.0, 2.0, 1.0, 2.0, 1.0] \\ \hline
9 & [8.0, 7.0, 1.0, 4.0] \\ \hline
\hline
\end{tabular}
\caption{A \textit{stemplot} of a grade distribution.}
\end{table}
\end{frame}

\begin{frame}[fragile]{Stemplots}
\small
Procedure:
\begin{enumerate}
\item Identify the approximate order of magnitude of the sample.
\item Within that order of magnitude, create $\approx 10$ \textit{stems,} corresponding to the base-10 digits.
\item For each data point, call the non-most significant digits the \textit{leaves} and drop the leaves in the category with the matching leaf.
\end{enumerate}
\textbf{Professor example:} What is the stemplot of 
\begin{verbatim}
[11, 22, 33, 44, 55, 66]
\end{verbatim}
\end{frame}

\begin{frame}[fragile]{Stemplots}
\small
Procedure:
\begin{enumerate}
\item Identify the approximate order of magnitude of the sample.
\item Within that order of magnitude, create $\approx 10$ \textit{stems,} corresponding to the base-10 digits.
\item For each data point, call the non-most significant digits the \textit{leaves} and drop the leaves in the category with the matching leaf.
\end{enumerate}
Let's create a stemplot of:
\begin{enumerate}
\item Our ages in MATH080
\item My age and the rest of my department
\end{enumerate}
(Stemplots lead in to the topic of histograms)
\end{frame}

\begin{frame}[fragile]{Histograms}
\alert{Histograms} are a tool for measuring \textit{probability distributions}.  The inputs are the data points and the corresponding relative frequencies, or plain frequencies. \\ \vspace{0.5cm}
\textbf{How many textbooks or books did you purchase for school last year?} (Type in the chat).
\begin{enumerate}
\item Determine the bins, or \textit{binning}
\item For each data point, drop it into the appropriate bin
\item Each time a measurement is dropped into a bin, the \textit{count} increases by 1.
\item If a histogram displays plain frequencies, it is called \textit{un-normalized.}
\item If a histogram displays relative frequencies, it is called \textit{normalized.}
\end{enumerate}
\end{frame}

\begin{frame}[fragile]{Histograms}
\begin{enumerate}
\item Histogram of books, by hand
\item Repeat with Excel/Calc
\end{enumerate}
Practice with the FREQUENCY function in Calc/Excel:
\begin{verbatim}
=FREQUENCY(A1:A99; B1:B11)
\end{verbatim}
Then press \alert{\textbf{control+shift+enter}} to execute on arrays of data and bins.  To \textit{normalize}, input the relative frequencies, or divide frequecies by $N$.  Assume the data is in C column:
\begin{verbatim}
=C1/N ...
\end{verbatim}
\end{frame}

\section{Conclusion}

\begin{frame}{Unit 0 Outline}
\begin{enumerate}
\item Topics from Chapter 1: 1.1, 1.2, 1.3
\begin{itemize}
\item What is a statistic?
\item Probability examples
\item Data and sampling
\end{itemize}
\item Topics from Chapter 2: 2.1 - 2.4, 2.5 - 2.8
\begin{itemize}
\item Data visualization
\item Location of the data in numerical space
\end{itemize}
\item Topics from Chapter 3: 3.1, 3.2, 3.3
\begin{itemize}
\item Two rules of probability
\end{itemize}
\end{enumerate}
\end{frame}

\end{document}
