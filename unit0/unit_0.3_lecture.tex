\documentclass{beamer}
\usetheme{metropolis}
\usepackage{graphicx}
\usepackage{amsmath}
\usepackage{tcolorbox}
\title{Elementary Statistics: Math 080}
\author{Jordan Hanson}
\institute{Whittier College Department of Physics and Astronomy}

\begin{document}
\maketitle

\begin{frame}{Unit 0 Outline}
\begin{enumerate}
\item Topics from Chapter 1: 1.1, 1.2, 1.3
\begin{itemize}
\item What is a statistic?
\item Probability examples
\item Data and sampling
\end{itemize}
\item Topics from Chapter 2: 2.1 - 2.4, 2.5 - 2.8
\begin{itemize}
\item Data visualization
\item Location of the data in numerical space
\end{itemize}
\item Topics from Chapter 3: 3.1, 3.2, 3.3
\begin{itemize}
\item Two rules of probability
\end{itemize}
\end{enumerate}
\end{frame}

\section{Topics from Chapter 3}

\begin{frame}[fragile]{Two Rules of Probability}
\alert{\textbf{The Multiplication Rule}}: If A and B are \textit{independent} events, then the probability
\begin{equation}
P(A~AND~B) = P(A) P(B)
\end{equation}
\alert{\textbf{The Addition Rule}}: If A and B are \textit{mutually exclusive} events, then the probability
\begin{equation}
P(A~OR~B) = P(A) + P(B)
\end{equation}
\textit{Independent} means knowledge that one event occurred does not change the probability of another event.  \textit{Mutually exclusive} means that the events cannot occur at the same time.
\end{frame}

\begin{frame}[fragile]{Two Rules of Probability}
\alert{\textbf{The Multiplication Rule}}: Example with coins. \\ \vspace{3cm}
\alert{\textbf{The Addition Rule}}: Example with coins.
\end{frame}

\begin{frame}{Two Rules of Probability}
Suppose you deal 4 cards from a 52 card playing deck (with four suits of 12 cards each) without replacing the cards.  What is the probability of obtaining four aces?
\begin{itemize}
\item A: 1 in 100
\item B: 1 in 2700
\item C: 1 in one million
\item D: 1 in 270,000
\end{itemize}
\end{frame}

\begin{frame}{Two Rules of Probability}
Suppose you deal 4 cards from a 52 card playing deck (with four suits of 12 cards each) without replacing the cards.  What is the probability of obtaining two hearts and two diamonds (any number for each)?
\begin{itemize}
\item A: 1 in 10
\item B: 1 in 33
\item C: 1 in 270
\item D: 1 in 3500
\end{itemize}
\end{frame}

\begin{frame}{Two Rules of Probability}
Suppose you deal 1 card from a 52 card playing deck (with four suits of 12 cards each) without replacing the card.  What is the probability of obtaining a heart or a diamond?
\begin{itemize}
\item A: 1 in 6
\item B: 1 in 3
\item C: 1 in 2
\item D: 1 in 30
\end{itemize}
\end{frame}

\section{Conclusion}

\begin{frame}{Unit 0 Outline}
\begin{enumerate}
\item Topics from Chapter 1: 1.1, 1.2, 1.3
\begin{itemize}
\item What is a statistic?
\item Probability examples
\item Data and sampling
\end{itemize}
\item Topics from Chapter 2: 2.1 - 2.4, 2.5 - 2.8
\begin{itemize}
\item Data visualization
\item Location of the data in numerical space
\end{itemize}
\item Topics from Chapter 3: 3.1, 3.2, 3.3
\begin{itemize}
\item Two rules of probability
\end{itemize}
\end{enumerate}
\end{frame}

\end{document}
